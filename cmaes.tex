The state-of-the-art evolutionnary algorithm CMA-ES is very well suited for these requirements.\\
The description of its architect, Nicolaus Hansen \color{blue}insert bibliography here\color{black} (see https://www.lri.fr/~hansen/cmaesintro.html) contains these words :
\begin{itemize}
	\item It is conceived to solve "difficult non-linear non-convex black-box optimisation problems in continuous domain". 
	\item It is feasible on "non-smooth and even non-continuous problems, as well as on multimodal and/or noisy problems".
	\item "The CMA-ES does not use or approximate gradients and does not even presume or require their existence"
	\item It is "competitive for global optimization".
\end{itemize}
In addition, it is extremely adaptive as only an initial standard deviation $\sigma$ and the population size $\lambda$ have to be tuned.\\
More information on CMA-ES can be found at https://www.lri.fr/~hansen/cmatutorial.pdf, especially in the parts \emph{0.3: Randomized Black-Box optimization} and \emph{5: Discussion}.\\
For further understanding, the reader can keep in memory that the algorithm samples a population $\Pi_{p}$ of $\lambda$ random points at iteration $p$. It then evaluates each one of them, and modifies its internal parameters so that the next $\lambda$ sampled points $\Pi_{p+1}$ will be more probably in the direction of the points of $\Pi_{p}$ that gave the smaller $\Phi$ values. It globally keeps memory of the fitness (objective function value) of the points it encountered.\\
\\
In what follows, the objective function $\Phi$ will also be called \emph{fitness function}.
