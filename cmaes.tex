The state-of-the-art evolutionnary algorithm CMA-ES is very well suited for these requirements.\\
Its architect, Nicolaus Hansen (see \cite{cmaintro}) describes it in these words :
\begin{itemize}
	\item It is conceived to solve "difficult non-linear non-convex black-box optimisation problems in continuous domain". 
	\item It is feasible on "non-smooth and even non-continuous problems, as well as on multimodal and/or noisy problems".
	\item "The CMA-ES does not use or approximate gradients and does not even presume or require their existence"
	\item It is "competitive for global optimization".
	\item It is adapted to search spaces of dimension between 3 and 100.
\end{itemize}
In addition, it is extremely adaptive as only an initial standard deviation $\sigma$ and the population size $\lambda$ have to be tuned.\\
A more precise description of CMA-ES can be found in \cite{cmaes} and \cite{cmatuto}, especially in the parts \emph{0.3: Randomized Black-Box optimization} and \emph{5: Discussion}.\\
For further understanding, the reader can keep in memory that the algorithm samples a population $\Pi_{p}$ of $\lambda$ random points at iteration $p$. It then evaluates each one of them, and modifies its internal parameters so that the next $\lambda$ sampled points $\Pi_{p+1}$ will be more probably in the direction of the points of $\Pi_{p}$ that gave the smaller $\Phi$ values. It globally keeps memory of the fitness (objective function value) of the points it encountered.\\
In what follows, the objective function $\Phi$ will also be called \emph{fitness function}.\\
\\
It is recommended to give the same sensitivity to the parameters i.e., in our case, to give the same range to the knobs (we rescaled each one of them to a 0-10 range before imputation to CMA-ES).\\
Furthermore, it is recommended to set $\sigma$ in the range $[0.2,0.5]$ times the size of the knobs range ($[2,5]$ in our case).\\
Finally, the recommended first value to give to $\lambda$ is $4+\lfloor3.log(\kappa)\rfloor$.

