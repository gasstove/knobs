\emph{For the following experiments, $\lambda=11$.}\\
\\
Fig. \ref{fig:umulevolution} below shows the evolution of the value of $J^{*}$ with $U_{mul}$.
\begin{figure}[h]
	\label{fig:umulevolution}
	\caption{Evolution of $J^{*}$ with $U^{mul}$}
	\includegraphics[width=7in]{figures/umul.png}
\end{figure}
As intuitively guessed, the more uncertainty i.e. freedom to the knobs there is, the smaller $J^{*}$ is.
More precisely, the tendencies observed are the following :
\begin{itemize}
	\item If $U^{add}$ is not too big (i.e. doesn't override the effect of $U^{mul}$), there is a minimum threshold on $U^{mul}$ for $J^{*}$ to be reasonably small : in our case, $U^{mul}$ around $25\% $ doesn't give satisfactory $J^{*}$ values ($35\% $ and $45\% $), while it enters in an acceptable range in comparison with the best results if it is bigger or equal to $50\% $. This shows the importance of the uncertainty, due to the lack of precision of the whole : a multiplicative uncertainty of $25\% $ or less is not enough.
	\item There is no real difference for the effect of $U^{mul}$ between $50\% and 75\% $: they are the typical executions.
	\item When the multiplicative uncertainty is very big, $U^{mul}=100\% $ (minimums are zero i.e. the physical minimums; and the maximums correspond to twice the partially monitored segments flow demands), the result seem to slightly improve. However, one run ($\sigma=5$ and $\frac{U^{add}}{\widetilde{F}}$ shows that it is not always the case :  the feasible space becomes very big and CMA-ES is not guided at all. In fact, as we will see observing the congestion pattern contour plots in this case, the results are not satisfactory.
\end{itemize} 
Below are 4 groups of two figures. They display, for each of the 4 values of $U^{mul}$, how the resulting congestion pattern fitting and the history of the knobs trough the execution evolve with $U^{add}$ and $\sigma$.\\
\begin{figure}
	\label{fig:umulcp25}
	\caption{$\mathbf{U_{mul}=25\%}$. Congestion pattern matching plots : evolution with $U_{add}$ and $\sigma$.}
	\includegraphics[width=7in]{figures/results_figures/Umul/cp_Umul_25_lambda_11.png}
\end{figure}	
\begin{figure}
	\label{fig:umulknobs25}
	\caption{$\mathbf{U_{mul}=25\%}$. Knobs history plots : evolution with $U_{add}$ and $\sigma$.}
	\includegraphics[width=7in]{figures/results_figures/Umul/knobs_Umul_25_lambda_11.png}
\end{figure}	
\\
\emph{$U^{mul}=25\% $}: Fig. \ref{fig:umulcp25}, with $U{mul}=25\% $, shows the effect on the result of excessively tight constraints.  With $U{add}=2.5\% $ and $5\% $ (two first columns), $\mathscr{C}$ cannot grow enough to fill $\widetilde{\mathscr{C}}$, which is responsible for a very high $E_{CP}^{*}$.\\
In the last column, $U_{add}=10\% $ gives a good space to all the knobs (many of them having their physical boundaries $\mathscr{B}$). In this case, $U{mul}=25\% $ has no effect, which is the reason for the good "typical" result obtained.\\
\emph{Conclusion}: as highlighted previously by Fig. \ref{fig:umulevolution}, setting $U^{mul}$ to a value smaller than a significant threshold ($25\% $ is quite high !) expels the good acceptable solutions from the feasible space.\\

There is nothing clear to conclude from the knobs evolution on Fig. \ref{fig:umulknobs25}., except that many end up on their boundaries (up to 8 of 12 !), which reflects that they were not adequately set i.e. that we describe a far-from-reality situation.
\\
\begin{figure}
	\label{fig:umulcp50}
	\caption{$\mathbf{U_{mul}=50\%}$. Congestion pattern matching plots : evolution with $U_{add}$ and $\sigma$.}
	\includegraphics[width=7in]{figures/results_figures/Umul/cp_Umul_50_lambda_11.png}
\end{figure}
\begin{figure}
	\label{fig:umulknobs50}
	\caption{$\mathbf{U_{mul}=50\%}$. Knobs history plots : evolution with $U_{add}$ and $\sigma$.}
	\includegraphics[width=7in]{figures/results_figures/Umul/knobs_Umul_50_lambda_11.png}
\end{figure}	
\\
\emph{$U^{mul}=50\% $}: We observe in Fig. \ref{fig:umulcp50} only typical results, with, as expected, a slight improvement while loosing the $U_{add}$ constraint.

In Fig. \ref{fig:umulknobs50}, there are more situations than previously with many knobs converging to another value than their  boundaries. We see a situation (middle-down figure) where the knobs did not finish converging, although the result is good : this illustrates the "equivalent results with different knobs" mentioned before. Some knobs whose action does not affect $\mathscr{C}$ can evolve in a correlated way (e.g. one on-ramp and one off-ramp knob both increase their value), maintaining all the contributions on a constant value. Their variations are therefore invisible to $J$ and $\mathscr{C}$.\\
\begin{figure}
	\label{fig:umulcp75}
	\caption{$\mathbf{U_{mul}=75\%}$. Congestion pattern matching plots : evolution with $U_{add}$ and $\sigma$.}
	\includegraphics[width=7in]{figures/results_figures/Umul/cp_Umul_75_lambda_11.png}
\end{figure}
\begin{figure}
	\label{fig:umulknobs75}
	\caption{$\mathbf{U_{mul}=75\%}$. Knobs history plots : evolution with $U_{add}$ and $\sigma$.}
	\includegraphics[width=7in]{figures/results_figures/Umul/knobs_Umul_75_lambda_11.png}
\end{figure}	
\\
\emph{$U^{mul}=75\% $}: Fig. \ref{fig:umulcp75} is similar to the preceding Fig. \ref{fig:umulcp50}. Unusual but acceptable congestion shapes that we did not observe before can appear when $U^{add}$ is small and $U^{mul}$  is large. 

There are many situations in Fig. \ref{fig:umulknobs75}. where many knobs (up to 9 of 12) converge far from their boundaries : this is a desirable situation and is expected, because we loosened the boundaries.\\
\begin{figure} 
	\label{fig:umulcp100}
	\caption{$\mathbf{U_{mul}=100\%}$. Congestion pattern matching plots : evolution with $U_{add}$ and $\sigma$.}
	\includegraphics[width=7in]{figures/results_figures/Umul/cp_Umul_100_lambda_11.png}
\end{figure}
\begin{figure}
	\label{fig:umulknobs100}
	\caption{$\mathbf{U_{mul}=100\%}$. Knobs history plots : evolution with $U_{add}$ and $\sigma$.}
	\includegraphics[width=7in]{figures/results_figures/Umul/knobs_Umul_100_lambda_11.png}
\end{figure}	
\\
\emph{$U^{mul}=100\% $}: Fig. \ref{fig:umulcp100} shows too loose constraints lead too unlikely congestion situations. In this case, all the knob boundaries are their physical boundaries from $\mathscr{B}$ or are close to them. CMA-ES is not "guided" by boundaries that reflect likely traffic, it therefore finds a way of decreasing the congestion error by making non-physic "wholes" in the congestion patterns.\\
As we began to observe in Fig. \ref{fig:umulcp75}, the worst cases happen when $U^{add}$ is small. In this cases (first and second column), most of the knobs boundaries are very loose except 4 of them, associated with small knob group flow demands, which always give tight boundaries with the multiplicative uncertainty.\\
Increasing $U^{add}$ diminishes this bad effect although it does not disappear : $U^{mul}=100\% $ is not a good choice.\\
This effect is not caused by the congestion threshold $d_{i}^{*}$ choice : full density or speed contour plots show the same non-physical undesirable effect as well. We see them in Fig. \ref{fig:badcontourplot}, a speed contour plot that intuitively reflects the congestion (located where the speeds are clearly slower).
We see in this figure that the traffic does not go smoothly from congested to non-congested, implying that this shapes are not due to $d_{i}^{*}$ but to truly non-physic phenomena output by BeATS.
\begin{figure}
	\label{fig:badcontourplot}
	\caption{Speed contour plot. Non-physical effects on that congestion that appear when $U^{mul}$ is too large (100 \%)}
	\includegraphics[width=7in]{figures/results_figures/Umul/badcontourplot.png}
\end{figure}