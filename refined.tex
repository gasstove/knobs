The local uncertainty described in \ref{sec:uncertainty} prevents us from keeping Eq. \ref{eq:balance} as a constraint for the parameters.\\
\\
$\forall i \in {\llbracket 1,\gamma \rrbracket},\ let\ the\ most\ permissive\ uncertainties:$\\
$\Delta_{i}^{-}=\max{\{|\Delta_{i}|-U^{add};|\Delta_{i}|.(1-U^{mul})\}}$\\
$\Delta_{i}^{+}=\min{\{|\Delta_{i}|+U^{add};|\Delta_{i}|.(1+U^{mul})\}}$\\
\\
Taking the local uncertainty into account in Eq. \ref{eq:balance} is translated into the following linear inequality constraint:
\begin{equation}
\label{eq:ineq}
	\forall i\in \llbracket 1,\gamma \rrbracket,\ \Delta_{i}^{-}\leq |\sum\limits_{j\in g_{i}}\sigma_{j}.k_{j}.\Theta| \leq \Delta_{i}^{+}
\end{equation}
\\
These $\gamma$ inequalities drastically reduce the size of the search space, defining a new \emph{feasible space}.\\
\\
\emph{Comments:}\\
We can now illustrate and justify the form that we have adopted for the uncertainty. This form allows us to quantify the freedom given to the result: the flow balance of each knob-group is between $(1-U^{mul})$ and $(1+U^{mul})$ times what has been measured by the mainline sensors, aknowledging that we don't accept less than $\pm U^{add}$ cars precision on the measures.\\
Taking into account $U^{add}$ is indispensable. This is observed in the case of single-knob groups, where Eq. \ref{eq:balance} leads to a unique value for the knob of the group. Let us call it \emph{perfect value of the knob i}, denoted $k_{i}^{perfect}$. It immediately follows from Eq. \ref{eq:ineq} that two new boundaries are set for $k_{i}$, if they are tighter than $[0,m_{i}]$ : $k_{i}\in [(1\pm U^{mul}).k_{i}^{perfect}]$\\
The problem happens when the perfect value is a ridiculously small quantity. The maximum obtained with $(1+ U^{mul}).k_{i}^{perfect}$ then corresponds often to a total daily flow of less than $500$ cars exiting the ramp, which is not acceptable.\\ 
\\
\emph{Example:} In the scenario we study, one of the ramps has a perfect value of $0.016$, which leads to a maximum of $0.032$ i.e. $406$ cars going through the ramp during the whole day if $U^{mul}$ is set to $100\%$ (very permissive: the daily flow can double what is measured by the mainline sensors). This is way too small for the scenario.\\ 
However, in fact, the fluctuation allowed by the new boundaries of this ramp is of $406\ cars$, which is $13$ times smaller than $U^{add}=5 \% .[mean\ on\ i\in T\ of\ \widetilde{F_{i}}]=5139\ cars$: the sensors do not have this level of precision, and the sensor noise/bias is responsible for this impossible perfect value. \\
Once $U^{add}$ is taken into account, the maximum of the knob becomes $0.42$, which corresponds to $5329\ cars$  and offers an acceptable range to the flow going through the ramp.\\