The local uncertainty described in \ref{sec:uncertainty} prevents us from keeping Eq. \ref{eq:balance} as a constraint for the parameters.\\
\\
$\forall i \in {\llbracket 1,\gamma \rrbracket},\ let\ the\ most\ permissive\ uncertainties:$\\
$\Delta_{i}^{-}=\max{\{|\Delta_{i}|-U^{add};|\Delta_{i}|.(1-U^{mul})\}}$\\
$\Delta_{i}^{+}=\max{\{|\Delta_{i}|+U^{add};|\Delta_{i}|.(1+U^{mul})\}}$\\
\\
Taking the local uncertainty into account in Eq. \ref{eq:balance} is translated into the following linear inequality constraint:
\begin{equation}
\label{eq:ineq}
	\forall i\in \llbracket 1,\gamma \rrbracket,\ \Delta_{i}^{-}\leq |\sum\limits_{j\in K\cap S_{i}}\sigma_{j}.k_{j}.T_{j}| \leq \Delta_{i}^{+}
\end{equation}
\\
These $\gamma$ inequalities drastically reduce the size of the search space, defining a new \emph{feasible space}.\\
\\
\emph{Comments:}\\
We can now illustrate and justify the form that we have adopted for the uncertainty. This form allows us to quantify the freedom given to the result: the flow balance of each knob-group is between $(1-U^{mul})$ and $(1+U^{mul})$ times what has been measured by the mainline sensors, aknowledging that we don't accept less than $\pm U^{add}$ cars precision on the measures.\\
Taking into account $U^{add}$ is indispensable. This is observed in the case of single-knob groups, where Eq. \ref{eq:balance} leads to a unique value for the knob of the group. Let us call it \emph{perfect value of the knob i}, denoted $k_{i}^{*}$. It immediately follows from Eq. \ref{eq:ineq} that two new boundaries are set for $k_{i}$, if they are tighter than $[0,m_{i}]$.  If the perfect value is a ridiculously small quantities, the maximum obtained with $(1\pm U^{mul}).k_{i}^{*}$ corresponds often to a total daily flow of less than $50$ cars exiting the ramp, which is not acceptable.\\ 
\\
\emph{Example:} One of the ramps has a perfect value of $0.02$, which leads to a maximum of $??$ cars going through the ramp during the whole day if $U^{mul}$ is set to $100\%$ (very permissive: the daily flow can double what is measured by the mainline sensors). This is too small for any scenario.\\ The fluctuation allowed by the new boundaries of this ramp is of $??\ cars$, which is ?? times smaller than $U^{add}$: the sensors do not have this level of precision, and the sensor noise/bias is responsible for this impossible perfect value. \\
Once $U^{add}$=$[10\%.(measured\ daily\ mainline\ flow)]$ is taken into account, the maximum of the knob becomes $0.7$, which corresponds to $??$ cars and offers an acceptable range to flow through the ramp.\\