The \emph{large traffic model} to calibrate is characterized by the following inputs and outputs:
\begin{itemize}
	\item \emph{Input:} 
	\begin{itemize}
		\item   duration of the scenario and time step.
		\item   value of the exit flow demand at every source and non-mainline sink for every time step.\color{red}
		\item	other parameters proper to the model, assumed to be already calibrated
	\end{itemize}
	\item \emph{Output:}
	\begin{itemize}
		\item   value of the exit flow on every link, for every time step.
		\item	value of the density on every link, for every time step.		
	\end{itemize}
\end{itemize}
\emph{Traffic model assumptions:} \color{red}Describe here the assumptions we've made on the traffic model:\color{black}
\begin{itemize}
	\item $\approx 0$ cars beginning and end of the day
	\item $\approx$no queue on the ramps (at least on the off-ramps). Partially justified by the fact that we only input flows that are inferior to the capacity of each ramp (physical box constraints).
	\item other assumptions I've not thought about
\end{itemize}
