Vehicle miles traveled is computed using the classic relative difference formula.\\
\\
Value computation from BeATS monitored mainline links output and PeMS data:
\begin{equation*}
	 VMT(\vec{k}^{(p)})=\sum_{i\in{T}}\sum_{t=0}^{24h/dt}f^{(p)}_{i}(t.dt)*L_{i}
\end{equation*}\color{red} define dt (5min) \color{black}\\

\emph{Reduction of the search space:} we present here a method used to reduce the search space size forcing the VMT value, thanks to an input pre-processing.\\
VMT is the result of a simple \emph{a priori} calculation that does not need the simulation to run, if we know the boundary conditions at 0:00am and 12:00pm. As exposed in \ref{subsec:model}, we assume that these conditions are $0\ cars$ on every link (the actual values are very small in comparison with the total daily vehicle miles traveled).\\
\\
Let $VMT^{a priori}(\vec{k}^{(p)})$ the expected VMT value computed from $\vec{k}^{(p)}$.\\
\\
$VMT^{a\ priori}(\vec{k}^{(p)})=VMT^{ref}+\sum \limits_{i\in K}\biggl[\sigma_{i}.T_{i}.k_{i}.\sum\limits_{\underset{j>i}{j\in T}}L_{j}\biggr]$\\
\\	
The \emph{a-priori} calculation above relies on anticipating the changes on VMT((1,...,1)) caused by changing the knobs from (1,...,1) to $\vec{k}^{(p)}$. For each ramp, the flow change is multiplied by the remaining mainline length and the "on/off-ramp indicator". All these contributions are then summed.\\
\\
This \emph{a priori} calculation empowers us to project the input $\vec{k}^{(p)}$ in order to ensure $VMT(\vec{k}^{(p)})\approx VMT^{PeMS}$.\\

\color{red} Where should the following part go? \color{black}\\
Furthermore, in addition to the contribution of $\Pi_{2}$, a penalization proportional to the distance traveled by the projection is added to the fitness function. This implements a project \& penalize process, for the same reasons as in \ref{subsec:multiple}.\\
\emph{Repairing:}\\
\\
The uncertainty $I^{global}$ acts on this projection: the knobs vector will be projected on the space between 2 hyperplans of  \text{\cursive S} defined by\\
$VMT^{PeMS}.(1-I^{global})=VMT^{apriori}(\vec{k}^{(p)})$ and\\
$VMT^{PeMS}.(1+I^{global})=VMT^{apriori}(\vec{k}^{(p)})$\\
and limited by the physical boundaries.
The projection is done with the following program:\\
\\
$minimize \ \ \ \ \ \norm{\vec{k}^{(p)}-\underline{\vec{k}}^{(p)}}_{2}$\\ \color{red}<- define $\underline{\vec{k}}^{(p)}$, clarify the three notations corresponding to the steps of the two input projections\color{black}\\
$s.t.$\\
\\
$VMT^{apriori}(\vec{k}^{(p)})>VMT^{PeMS}.(1-I^{global})$\\
$VMT^{apriori}(\vec{k}^{(p)})<VMT^{PeMS}.(1+I^{global})$\\
$k^{(p)}\in{[\vec{0},\vec{m}}]$\\
\\
\emph{Penalizing:}\\
\\
The penalization is proportional to the distance between the original point and projected point, normalized by the same \emph{order of magnitude} as in \ref{subsec:multiple}.



$VMT^{PeMS}$ is the value computed from the PeMS data. \color{red}<- We should think about when to introduce this notation and, more generally, the performance calculators value notation \color{black}
\\
\begin{equation*}
	\Pi_{2}(\vec{k}^{(p)})=100.\frac{|VMT(\vec{k}^{(p)})-VMT^{PeMS}|}{VMT^{PeMS}}+\Pi_{2}^{'}
\end{equation*}
\\
\\