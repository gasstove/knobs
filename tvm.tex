This quantity is the sum of the distance traveled on the mainline by each car, over the whole duration.
Obviously, it is computed using only the monitored mainline links, for the comparison with the data to be relevant.\\
\\
VMT computation on monitored mainline links output:
\begin{equation*}
	 VMT(\vec{k})=\sum_{i\in{T}}L_{i}\sum_{t\in \tau}f_{i}(t)
\end{equation*}
\\
Denoting $\widetilde{VMT}$ the value computed from the data using the same formula, the error is the relative difference :
\begin{equation*}
	E_{VMT}(\vec{k})=\frac{|VMT(\vec{k})-\widetilde{VMT}|}{\widetilde{VMT}}
\end{equation*}
~\\
\emph{Reduction of the feasible space:} we present here a method used to reduce the feasible space size by forcing the knobs to match the correct VMT value.\\
VMT is the result of a simple \emph{a priori} calculation that does not need the traffic model output computation, if we know the boundary conditions at $t=0$ and $t=D$. As exposed in \ref{subsec:traffic_model}, we assume that these conditions are $\approx 0\ cars$ on every link ($D$ has to be big enough for these conditions to be very small in comparison with the total number of vehicles going through the freeway during $D$).\\
\\
Let $VMT^{ref}=VMT(\vec{k}^{ref})$, a certain $VMT$ reference value output by the model. We suppose that $\vec{k}^{ref}$ is some feasible knobs vector (in our case, we used $\vec{k}^{ref}=(1,...,1)$).\\
Denoting $VMT^{a priori}(\vec{k})$ the expected VMT value computed from $\vec{k}$:\\
\begin{equation}
	\label{eq:TVMapriori}
	VMT^{a\ priori}(\vec{k})=VMT^{ref}+\mathlarger{\sum\limits_{i\in K}}\biggl[\sigma_{i}.k_{i}.\Theta.	\sum\limits_{\underset{j>i}{j\in T}}L_{j}\biggr]
\end{equation}	
The \emph{a-priori} calculation above consists in anticipating the deviation from $VMT^{ref}$ caused by changing the knobs from $\vec{k^{ref}}$ to $\vec{k}$. For each knob, the flow change resulting from its modification is multiplied by the remaining mainline length. These $\kappa$ contributions are then summed.\\
\\
The linear equation \ref{eq:TVMapriori} empowers us to constrain the input $\vec{k}$ in order to ensure\\ $\widetilde{VMT}=VMT^{a\ priori}(\vec{k})(\approx VMT(\vec{k}))$, thus reducing the size of the feasible space by one dimension:
\begin{equation*}
	\mathlarger{\sum\limits_{i\in K}}\biggl[\sigma_{i}.k_{i}.\Theta.	\sum\limits_{\underset{j>i}{j\in T}}L_{j}\biggr]+VMT^{ref}=\widetilde{VMT}
\end{equation*}
However, the global uncertainty applied on VMT (which is a global quantity computed from the sum over the whole time and space) forces us to loosen this last constraint equation.\\
Denoting $\widetilde{VMT}^{-}=\widetilde{VMT}.(1-U^{global})$ and $\widetilde{VMT}^{+}=\widetilde{VMT}.(1+U^{global})$, it becomes:\\
\begin{equation}	
	\label{eq:TVMineq}
	\widetilde{VMT}^{-}\leq\mathlarger{\sum\limits_{i\in K}}\biggl[\sigma_{i}.k_{i}.\Theta.	\sum\limits_{\underset{j>i}{j\in T}}L_{j}\biggr]+VMT^{ref}\leq \widetilde{VMT}^{+}
\end{equation} 