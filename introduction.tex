\IEEEPARstart{T}{his} paper solves a part of the large traffic model calibration problem. By calibrating such model, we mean that we try to reproduce as accurately as possible all the traffic phenomena on the freeway, while giving it an input as close to reality as possible (two goals than can be contradictory).\\
Large traffic models possess many input parameters, especially road characteristics and flow demands on each ramp. Our method partially solves the problem of the flow demands imputation.\\
On an incompletely monitored freeway scenario, we input to the monitored ramps the flows given by the measurements. We then associate to each non-monitored ramp a custom normalized flow profile called \emph{template} and an intensity coefficient called \emph{knob}. The input flow to the ramp will be the template multiplied by the knob. The method tries to match the mainline (freeway, by opposition to ramps) measurements in terms of congestion location and times and 2 other performance metrics, tuning only the values of the knobs of the non-monitored ramps.\\
The method is just the first-step in a wider context where the template shapes will also be modificable. Later, the calibration of other parameters (such as the \emph{fundamental diagrams}, characteristics of the mainline links) will be in a loop with the flow calibration, in order to calibrate the model as a whole.\\
This method, although being very formalized in this paper, is simple and intuitive. This is the report on an empirical progressive construction.
	
