For every monitored source or sink (except the mainline sink), we input the measured flow to the model as exit flow demand.\\ The assumptions made in \ref{subsec:traffic_model} imply that this demand is approximately equal to the actual flow going through the ramp, for all times and above mentioned links:\\
\begin{equation}
	\label{eq:noqueue}
	\forall i\in R\backslash K,\ \forall t\in \tau,\ \widetilde{r_{i}}(t)=\bar{r_{i}}(t)\approx {r_{i}}(t)\\
\end{equation}
Therefore, the only missing parameters to the model are the flow demand profiles of the non monitored ramps: $(\bar{r_{i}}(t))_{\stackrel{i\in K}{t \in{\tau}}}$.
Our method consists in mapping these $\kappa$ flow profiles into one parameter each.\\
To do that, a flow profile called \emph{template} is built for every non-monitored ramp. These templates, denoted $(t_{i}(t))_{\stackrel{i\in{K}}{t \in{\tau}}}$, consist in a normalized flow profile: a flow value is given to each element of $\tau$ and the resulting profile is normalized to a reasonable value $T$.
For each of the non-monitored ramps $i$, we define a multiplicative factor $k_{i}$ called \emph{knob} that will set the intensity of the template. 
That is, we input as exit flow demand of the ramp its corresponding template multiplied by the ramp knob : $k_{i}.t_{i}(t)$.\\
\\
The parameters of our imputation problem are therefore the \textbf{$\kappa$ knobs}, corresponding to the $\kappa$ non-monitored ramps.

In addition, due the same assumptions that gave Eq. \ref{eq:noqueue}, we have :\\
\\
$\forall i \in K,\ \forall t\in \tau,\ k_{i}.t_{i}(t)=\bar{r_{i}}(t)\approx r_{i}(t)$\\
$and,\ especially, with\ \Theta=\sum\limits_{t\in\tau}t_{i}(t):$\\
\\
\begin{equation}
	\label{eq:noqueuetemplate}
	\sum\limits_{t \in \tau} r_{i}(t)=k_{i}.\Theta\\
\end{equation}