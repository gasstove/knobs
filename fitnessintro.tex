The calibration method consists in minimizing jointly the three errors described in \ref{subsec:pcs_intro}. We accomplish this goal by minimizing an \emph{objective function} $\Phi$, which is the weighted sum of the three errors, in percentage :
	\begin{displaymath}
		\Phi:
		\left|
  		\begin{array}{rcl}
    	\mathscr{B} & \longrightarrow &[0,100] \\
    	\vec{k} & \longmapsto & w_{1}.E_{VHT}(\vec{k})+w_{2}.E_{VMT}(\vec{k})+w_{3}.E_{CP}(\vec{k}) \\
  	\end{array}
	\right.
	\end{displaymath}
$with\ w_{1}+w_{2}+w_{3}=100\ and\ \forall i\in \{1,2,3\},\ w_{i}\geq 0.$\\
\\
\emph{Note : the errors can lead to values superior to 1 thus giving values of $\Phi$ superior to 100\% but, to simplify, we will ignore these cases that are very far from the objective.}\\
\\
\emph{Uncertainty handling :} $I^{global}$, described in \ref{subsec:data}, defines a tolerance threshold for the error results. the error results below $I^{global}$ are set to zero, in order to avoid any differentiation between them (we do not have a level of precision below $I^{global}$).
We call \emph{contributions} the final contribution of every performance calculator in the sum composing the fitness function: the value in percentage multiplied by its weight, once the preceding tolerance has been applied.\\
\\ 
$Let\ (\rho_{i})_{i\in\llbracket 1,4 \rrbracket}\ the\ weights.$\\
$We\ have:\ \sum_{i=1}^{4} \rho_{i}=1\ and\ \forall i \in,\ \llbracket 1,4 \rrbracket,\ \rho_{i}>0$\\
\begin{displaymath}
		\Phi:
		\left|
  		\begin{array}{rcl}
    	\text{\cursive S} & \longrightarrow &[0,100] \\
    	\vec{k}^{(p)} & \longmapsto & \sum_{i=1}^{4}\rho_{i}.\Pi_{i}(\vec{k}^{(p)}).\mathds{1}_{(i=1)\cup(\Pi_{i}(\vec{k}^{(p)})>I^{global})}\\ 
  		\end{array}
		\right.\\
\end{displaymath}
This definition as a weighted sum of percentages implies that the values of $\Phi$ can be interpreted as a \emph{global error percentage}. The weight given to each component is equivalent to the importance we want to give to each one of them. In fact, due to the very good performance of CMA-ES giving a significant weight to each one is enough to ensure that the result will be the best possible \color{red}(<- this is bullshit)\color{black}.\\
Below is a description of the computation of each component.
