The calibration method consists in minimizing jointly the three errors described in \ref{subsec:pcs_intro}. We accomplish this goal by minimizing an \emph{objective function} $\Phi$, which is the weighted sum of the errors modified by the global uncertainty, as explained below.\\
\\
\emph{Uncertainty handling :} $I^{global}$, described in \ref{subsec:data}, defines a tolerance threshold for the error results. The error results below $I^{global}$ are set to zero, in order to avoid any discrimination between them (we do not have a level of precision below $I^{global}$).\\
Each of the three errors $E$ will therefore be multiplied by $\mathds{1}_{(E>I^{global})}.$\\ 
\\
Let $(w_{i})_{i\in\llbracket 1,3 \rrbracket}$ the weights, verifying :\\
$w_{1}+w_{2}+w_{3}=100\ and\ \forall i\in \{1,2,3\},\ w_{i}\geq 0.$\\
\\
Let the three error \emph{contributions}:\\
$\phi_{VHT}(\vec{k})=w_{1}.E_{VHT}(\vec{k}).\mathds{1}_{(E_{VHT}>I^{global})}$\\
$\phi_{VMT}(\vec{k})=w_{2}.E_{VMT}(\vec{k}).\mathds{1}_{(E_{VMT}>I^{global})}$\\
$\phi_{CP}(\vec{k})=w_{3}.E_{CP}(\vec{k}).\mathds{1}_{(E_{CP}>I^{global})}$\\
\\
$\Phi$ is defined by:\\
\begin{displaymath}
		\Phi:
		\left|
  		\begin{array}{rcl}
    	\mathscr{B} & \longrightarrow &[0,100] \\
    	\vec{k} & \longmapsto &  \phi_{VHT}(\vec{k})+\phi_{VMT}(\vec{k})+\phi_{CP}(\vec{k}) \\
  	\end{array}
	\right.
\end{displaymath}
\\
\emph{Note : the errors can lead to values superior to 1 thus giving values of $\Phi$ superior to 100\% but, to simplify, we will ignore these cases that are very far from the objective.}\\
\\

This definition as a weighted sum of percentages implies that the values of $\Phi$ can be interpreted as a \emph{global error percentage}. Thanks to the normalization of the errors, the weight given to each component is equivalent to the importance the operator wants to give to each one of them.
