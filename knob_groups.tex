We define here objects and notation to describe a simple situation: the knobs are closely monitored by nearby mainline sensors, leading often to a situation where a ramp is the only non-monitored ramp between two mainline sensors. This is equivalent to it being monitored, if it was not for the uncertainties.\\
\\
We call \emph{segment} the set of links between two consecutive mainline sensors, including the links containing these sensors. \\
We call \emph{knob group} each set of knobs whose corresponding ramp is connected to the same monitored segment.
This definition is illustrated in Fig. \ref{fig:scheme}.\\
We call \emph{partially monitored segment} the monitored segments associated with a knob group i.e. containing at least one non-monitored ramp.\\
\\
Denoting $\gamma$ the total number of knob groups, we have:\\
\\
$Partially\ monitored\ segments:\ (S_{i})_{i \in \llbracket 1,\gamma \rrbracket}$\\ 
\\
Formal definition:\\
$\exists\ !\ \gamma\in M,\ \exists\ !\ ((\beta_{i},\eta_{i}))_{i\in\llbracket 1,\gamma\rrbracket}\in (T^2)^{\llbracket 1,\gamma\rrbracket}\ s.t.,$\\
$denoting\ S_{i}=\llbracket \beta_{i},\eta_{i} \rrbracket\ and\ S=\underset{i\in \llbracket 1,\gamma \rrbracket}{\bigcup}  S_{i}:$\\
$\forall i\in\llbracket 1,\gamma \rrbracket,$
\begin{enumerate}
	\item $\llbracket \beta_{i},\eta_{i} \rrbracket \cap T=\{\beta_{i},\eta_{i}\}$
	\item $\llbracket \beta_{i}, \eta_{i} \rrbracket \cap K\not= \{\emptyset \}$
\end{enumerate}
$and\ \forall k\in K,\ k\in S.$\\
\\
\\
$Knob\ groups:\ (g_{i})_{i\in \llbracket 1,\gamma \rrbracket}$\\
\\
Formal definition:\\
$\forall i \in \llbracket 1, \gamma \rrbracket,\ g_{i}=\llbracket \beta_{i}, \eta_{i} \rrbracket \cap K $\\
\\
We can deduce the value of the daily flow brought by the knobs of each group from the \emph{knob group flow balance}: the difference between all the flows entering and all the flows exiting their incomplete monitored segment. That is the sum of the flow exiting the mainline entrance of the segment and the flows exiting the monitored on-ramps throughout the segment minus the sum of the flow exiting the mainline exit of the segment and the flows exiting the monitored off-ramps throughout the segment. \\ 
\color{red}Put here the paragraph Gabriel wrote\color{black}\\
\\
$Knob\ group\ flow\ balances:\ (\Delta_{i})_{i\in \llbracket 1,\gamma \rrbracket} $\\
\\
Formal definition:\\
$\forall i \in \llbracket 1,\gamma \rrbracket,\ \forall t\in \tau,$\\
$\Delta_{i} =$\small $\mathlarger{\sum\limits_{t\in \tau}}\bigg[\widetilde{f}_{\beta_{i}}(t)-\widetilde{f}_{\eta_{i}}(t)+\sum\limits_{j\in (R\backslash K)\cap S_{i}}\sigma_{j}.\widetilde{r_{j}}(t)\bigg]$\normalsize 
\\
\\
\\
The balance equation of each partially monitored segment is:\\
\\
$0=\mathlarger{\sum\limits_{t\in \tau}}\bigg[f_{\beta_{i}}(t)-f_{\eta_{i}}(t)+\sum\limits_{j\in R\cap S_{i}}\sigma_{j}.r_{j}(t)\bigg]$\\
\\
As stated in eq. \ref{eq:noqueue}, the model-output flows exiting the ramps are equal to their demand flows. The balance equation becomes therefore:\\
\\
$0=\Delta_{i}+\mathlarger{\sum\limits_{t\in \tau}}\bigg[\sum\limits_{j\in K\cap S_{i}}\sigma_{j}.r_{j}(t)\bigg]$\\
\\
Leading to:\\
\\
$\Delta_{i}=\mathlarger{-\sum\limits_{t\in \tau}}\bigg[\sum\limits_{j\in K\cap S_{i}}\sigma_{j}.r_{j}(t)\bigg]$\\
\\
$\Leftrightarrow \Delta_{i}=\mathlarger{-\sum\limits_{j\in K\cap S_{i}}}\sigma_{j}\bigg[\sum\limits_{t\in \tau}r_{j}(t)\bigg]$\\
\\ 
and thanks to eq. \ref{eq:noqueuetemplate}:
\begin{equation}
\centering
\label{eq:balance}
\Leftrightarrow\ \ \Delta_{i}=-\sum\limits_{j\in K\cap S_{i}}\sigma_{j}.k_{j}.\Theta
\end{equation}
Eq. \ref{eq:balance} shows that, for every knob group, the knobs composing it are linked by one linear equation.\\
This equation determines uniquely the value of the single-knob groups and links the multiple-knob groups with one linear constraint. The next paragraph describes how we apply uncertainties to this equation in order to produce new, closer to reality knob boundaries.\\

