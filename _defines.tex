
\newcommand{\slfrac}[2]{\left.#1\middle/#2\right.}

\def\onetonminusone{\{1\hdots n-1\}}
\def\oneton{\{1\hdots n\}}
\def\twoton{\{2\hdots n\}}

\def\Gammaoneton{\Gamma_{1\hdots n}}
\def\gammaoneton{\gamma_{1\hdots n}}
\def\muoneton{\mu_{1\hdots n}}
\def\sigmasqoneton{\sigma^2_{1\hdots n}}
\def\barGammaoneton{\bar\Gamma_{1\hdots n}}


\def\Ao{A^o}
\def\As{A^*}
\def\Bo{B^o}
\def\Bs{B^*}

\def\intCoverTwo{[-\nicefrac{C}{2},\nicefrac{C}{2}]}
\def\modx{\text{ mod}^*}
\newtheorem{theorem}{Theorem}[section]
\newtheorem{lemma}[theorem]{Lemma}
\newtheorem{proposition}[theorem]{Proposition}
\newtheorem{corollary}[theorem]{Corollary}
\newtheorem{remark}[theorem]{Remark}

\newcommand{\refsec}[1]{\ref{#1}}
\newcommand{\Srefsec}[1]{Section \ref{#1}}
\newcommand{\srefsec}[1]{Sec. \ref{#1}}
\newcommand{\Srefsecs}[1]{Sections \ref{#1}}
\newcommand{\srefsecs}[1]{Secs. \ref{#1}}

\newcommand{\refeq}[1]{(\ref{#1})}
\newcommand{\Erefeq}[1]{Equation (\ref{#1})}
\newcommand{\erefeq}[1]{Eq. (\ref{#1})}
\newcommand{\Erefeqs}[1]{Equations (\ref{#1})}
\newcommand{\erefeqs}[1]{Eqs. (\ref{#1})}

\newcommand{\reffig}[1]{\ref{#1}}
\newcommand{\freffig}[1]{Figure \ref{#1}}
\newcommand{\Freffig}[1]{Figure \ref{#1}}
\newcommand{\freffigs}[1]{Figures \ref{#1}}
\newcommand{\Freffigs}[1]{Figures \ref{#1}}

\newcommand{\reftab}[1]{\ref{#1}}
\newcommand{\treftab}[1]{Table \ref{#1}}
\newcommand{\Treftab}[1]{Table \ref{#1}}
\newcommand{\treftabs}[1]{Tables \ref{#1}}
\newcommand{\Treftabs}[1]{Tables \ref{#1}}

% IEEE FLOATS --------------------------------------------


% An example of a floating figure using the graphicx package.
% Note that \label must occur AFTER (or within) \caption.
% For figures, \caption should occur after the \includegraphics.
% Note that IEEEtran v1.7 and later has special internal code that
% is designed to preserve the operation of \label within \caption
% even when the captionsoff option is in effect. However, because
% of issues like this, it may be the safest practice to put all your
% \label just after \caption rather than within \caption{}.
%
% Reminder: the "draftcls" or "draftclsnofoot", not "draft", class
% option should be used if it is desired that the figures are to be
% displayed while in draft mode.
%

% An example of a floating table. Note that, for IEEE style tables, the
% \caption command should come BEFORE the table. Table text will default to
% \footnotesize as IEEE normally uses this smaller font for tables.
% The \label must come after \caption as always.
%
\newcommand{\ieeetable}[4]{
    \begin{table}[!t]
    \renewcommand{\arraystretch}{1.3}
     if using array.sty, it might be a good idea to tweak the value of
     \extrarowheight as needed to properly center the text within the cells
    \caption{#2}
    \label{#3}
    \centering
    % Some packages, such as MDW tools, offer better commands for making tables
    % than the plain LaTeX2e tabular which is used here.
    \begin{tabular}{#1}
    \hline
    One & Two\\
    \hline
    Three & Four\\
    \hline
    \end{tabular}
    \end{table}
}
