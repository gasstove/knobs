Three performance calculators (PC) are used for the error calculation:
\color{red}\emph{We describe here qualitatively the performance calculators themselves, not the error calculation (that's why I mention congestion and not congestion pattern fitting).}\color{black}
\begin{itemize}
	\item \emph{Vehicle miles traveled} (VMT): sum of the distance travelled on the mainline by each car, over the whole day. \emph{Explain why it is important in traffic study, why we have chosen it} 
	\item \emph{Vehicle hours traveled} (VHT): sum of the time spent on the mainline by each car, over the whole day. \emph{explain why it is important in traffic study, why we have chosen it} 
	\item \emph{Congestion}: according to \emph{blabla} theory, we define the congested links as the links where the density exceeds the \emph{critical density} defined by the link's fundamental diagram. The main feature of our method is to fit the locations and times of these congested links to what the measurements indicate.
\emph{explain why we have chosen it}
\end{itemize}
