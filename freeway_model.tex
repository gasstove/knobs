The freeway model is composed by the physical characteristics of the freeway.
We consider a one-way segment of freeway and its on-ramps and off-ramps.\\
We define here the components of the considered freeway model :
\begin{itemize}
	\item \emph{Mainline}: The freeway itself i.e. the central part where the cars go fast.
	\item \emph{Ramps}: The portions of road connected to the mainline that allow to enter or exit it.
	\item \emph{On-ramp}: Ramp to enter the mainline.
	\item \emph{Off-ramp}: Ramp to exit the mainline.
	\item \emph{Link}: a link is a segment of freeway or a ramp. The links are separated by \emph{nodes}. Each ramp is connected to a node and a node can only be connected to one ramp. There can be several consecutive mainline links without ramps.
	\item \emph{Linear order}: Links ordered accordingly to the traffic direction.
	\item \emph{Number of lanes of a link}: number of cars than can be side by side on the same level of the link.
	\item \emph{Topography of the freeway}: The length of the links and their number of lanes.
	\item \emph{Source}: Link which is an on-ramp or the entry of the mainline.
	\item \emph{Sink}: Link which is an off-ramp or the exit of the mainline.
	\item \emph{Monitored link}: Link which possesses a fully functional sensor monitoring all of its lanes in terms of flow. If it is a mainline link, it must also be monitored in terms of density (or speed, which are equivalent).
\end{itemize}


