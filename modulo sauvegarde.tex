This section reduces the periodic problem to the one-cycle problem.\\
Untill now, we considered that every signal had only one green light interval of time. In order to solve the general C-periodic problem, we only have to focus on one hypercube with $C$ sides. Therefore, we can replace the periodic objective function by any function that would be equal on an hypercube with $C$ sides and only look at solutions on that hypercube. Let us choose $\mathbb{H} = [-C/2, C/2[^n$ for example.\\
By linearity, we can deal with each bandwidth function separately : let us call $b_P$ the bandwidth function for a single cycle as intruduced earlier and $b^*_P$ the periodic bandwidth function. \\
Let $I \subset [|1,n|]$ be the set of intersection crossed by $P$, let $(e_1,...,e_n)$ be the canonical base of $\Omega = \mathbb{R}^{[|1,n|]}$. Let $V = vect({e_i, i \in I})$ and $W = vect({e_i, i \in [|1,n|]\backslash I})$. Let $e = \sum\limits_{i=1}^{n}e_i$. \\
\begin{lemma}
$\forall \omega \in \Omega$,\\
$b^*_P(\omega) = \sum\limits_{k \in K} b_P(\omega + Ck)$,
where $K = \{\sum\limits_{i \in I}z_i e_i, z_i \in \mathbb{Z}\} - \mathbb{Z}e$
\end{lemma}
\begin{proof}
First, $\forall \omega \in \Omega, \forall k \in K,$
\begin{equation}
b^*_P(\omega) \geq b_P(\omega + Ck)
\end{equation}
because the left term consider one particular cycle for every green light while the right term consider all of them. This prooves the lemma for the case $b^*_P(\omega) = 0$ \\ \\
Then $\forall \omega \in \Omega$,
\begin{equation*}
b_P(\omega) > 0 \Rightarrow (\forall k \in K-\{0\},~b_P(\omega + Ck) = 0)
\end{equation*}
because if we take two green light interval, both shorter than $C$, with a non-empty inersection, and we shift them of two different integer numbers of times the cycle $C$, they must have an empty intersection.\\
Therefore, $\forall \omega \in \Omega$, there exists at most one $k \in K$ such that $b_P(\omega + Ck) > 0$. \\ \\
As we built $b_P$ for one specific cycle, $\exists \Delta_P \in \Omega,~\forall \omega \in \mathbb{H}+\Delta_P,$
\begin{equation}
\label{eq:hypercube}
b_P(\omega) = b^*_P(\omega)
\end{equation}
Also, $\forall \omega \in \Omega,~\exists! z \in \mathbb{Z}^{[|1,n|]}$,
\begin{equation*}
(\omega + C z) \in \mathbb{H}+\Delta_P
\end{equation*}
with $z = z_V + z_W$, $z_V \in V$ and $z_W \in W$. \\
Now,
\begin{equation}
b^*_P(\omega) = b^*_P(\omega + Cz)
\end{equation}
because $b^*_P$ is periodic
\begin{equation}
b^*_P(\omega) = b_P(\omega + Cz)
\end{equation}
because of the equation (\ref{eq:hypercube})
\begin{equation*}
b^*_P(\omega) = b_P(\omega + z_V C + z_W C)
\end{equation*}
\begin{equation}
b^*_P(\omega) = b_P(\omega + z_V C)
\end{equation}
Because changing signals offset on an uncrossed intersections does not change the bandwidth of the path $P$. \\
-Either $z_V = \lambda e$,~$\lambda \in \mathbb{R}$, and then $b_P(\omega +z_V C) = b_P(\omega)$ because shifting all the green ligths intervals of the same displacement will not change their intersection.\\
-Or $z_V \in K$.\\
In both cases, if $b^*_P(\omega) > 0$ we can add all the others terms, equals to zero:\\
\begin{equation*}
b^*_P(\omega) = \sum\limits_{k \in K} b_P(\omega + Ck)
\end{equation*}
which finishes to proove the lemma.
\end{proof}
As we only want to look at the objective function on $\mathbb{H}$, we can eliminate all the terms which supports don't intersect $\mathbb{H}$. Their are at most $2^{|I|}-1$ of them.\\
Note : This number might seem huge but the experience shows that there is usually a lot less of usefull terms and it is very easy to find them.\\
From now on, we will consider all the terms as different paths and come back to our previous single-cycle approach.