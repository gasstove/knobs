We define the congested links as the links where the density exceeds some \emph{critical density} deduced from the traffic model. The main feature of our calibration method is to fit the locations and times of these congested links to what the measurements indicate.\\
\\
We call \emph{contour plot} the graph representing the value of a quantity on every mainline link at all times: the mainline links as absciss and time steps as ordinates.
Below is an example of a density contour plot for one day on a 135 links freeway (time is descending).\\
%\begin{figure}
%	\caption{Example of a contour plot}
%	\includegraphics[width=1in]{figures/contourexample.eps}
%\end{figure}
This plot is used to monitor easily where and when the congestion is : here, it is contained in the framed part. In what follows, by analogy, we will call \emph{density contour plot} the set $\big\{d_{i}(t)|i\in M,\ t\in\tau\big\}$ and \emph{pixel} each of its elements (i.e. each value of the density).

			\begin{equation*}
				CP(p)=\sum_{t\in{\frac{24h}{dt}}}\sum_{l\in{L}}\mathds{1}_{wrong\ congestion\ state\ links}
			\end{equation*}
Relative difference : $\tau_{p}(CP)=100* \frac{CP(p)}{Total\ area\ of\ the\ boxes}$
