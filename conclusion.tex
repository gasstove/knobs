Calibration is one of the main concerns regarding the viability of macroscopic large traffic models. It consists in matching the reality of traffic with credible -if possible real- data, in order to be able to realize further experiments with the model (e.g. predict and quantify the decongestion effect of adding an off-ramp at some point of a freeway). In this paper, we expose a method to calibrate the missing input total daily flows, given their shape as a \emph{"template"}.\\
\\
To do so, we first formalize the calibration process as a general black-box optimization problem. We then choose a numerical method that is efficient and relevant for any problem : the powerful CMA evolution strategy.\\
We apply this numerical method to our particular freeway, large traffic simulator and template choice. The experiments show tendencies with the variation of the parameters (mainly the uncertainty). However, the inaccurate template choice, that greatly limits the best result quality, does not allow to describe deeply and accurately the effects of the parameters, especially the initial standard deviation and population size. \\
\\
This paper is a report on the advancement of a work that is much wider. It will allow to calibrate jointly all the parameters of large traffic models and will be applicable to any of such, with any scenario.