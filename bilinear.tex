$\mathcal{P}$ is the set of all possible paths on the arterial. As a path is determined by the entry and the exit, $|\mathcal{P}| = (2n+2)(2n+1) - 2n$ as it is the number of possible choices of two ordered elements in a set of cardinal $2n+2$ minus all the paths corresponding to crossing the arterial following a side street (cf Figure \ref{fig:intersection}). For all $P \in \mathcal{P}$, let $w_P$ be the weight given to the path $P$ as said at the beginning of subsection \ref{subsec:formulation}.\\ \\
The function we want to maximize in our problem is:
\begin{align}
f_{\mathcal{P}}  (\mathbf{\omega}) = \sum_{P \in \mathcal{P}} w_{P} b_P(\omega)
\end{align}
which can be writen
\begin{align}
f_{\mathcal{P}} (\mathbf{\omega}) = \underset{P = (p_1,...,p_k) \in \mathcal{P}}{\sum w_P \; \max}(\underset{i,j\in\{1,...,k\}}{0,~ \min}(\omega_{p_i}-\omega_{p_j}+g_{p_i,p_j}))
\end{align}
$f_\mathcal{P}$ is non-convex and non-differentiable. Therefore, conventional convex or gradient methods are not going to help us. In what follows, we construct a proxy Mixed Binary Linear Program which solution will also solve $P$.\\
Here is a representation of the differents intermediate problems and the different steps we are going to follow in order to obtain this simpler problem:\\
\begin{center}
\begin{tabular}{c}
Non-convex problem\\
$\downarrow$\\
A set of Linear Programs\\
$\downarrow$\\
Mixed Binary Bilinear Program\\
$\downarrow$\\
Mixed Binary Linear Program\\
\\
\end{tabular}
\end{center}
First, we define $\forall \mathcal{U} \subset \mathcal{P}$, $\forall \omega \in \Omega$, 
\begin{align}
f_{\mathcal{U}} (\omega) = \sum_{P \in \mathcal{U}} w_P b_P(\omega)
\end{align}
or again
\begin{align}
f_{\mathcal{U}} (\mathbf{\omega}) = \underset{P = (p_1,...,p_k) \in \mathcal{U}}{\sum w_p ~ \max}(\underset{i,j\in\{1,...,k\}}{0,~ \min}(\omega_{p_i}-\omega_{p_j}+g_{p_i,p_j}))
\label{eq:sousproblem}
\end{align}
Now, we define $f_{\mathcal{U},r} (\omega)$ by removing the lower bounds from (\ref{eq:sousproblem}),
\begin{align}
f_{\mathcal{U},r} (\mathbf{\omega}) = \underset{P = (p_1,...,p_k) \in \mathcal{U}}{\sum ~~~ w_P} ~\underset{i,j\in\{1,...,k\}}{\min~(\omega_{p_i}}-\omega_{p_j}+g_{p_i,p_j})
\end{align}
$f_{\mathcal{U},r}$ is concave (being the sum of concave functions) and defines a linear function.

We will now connect these relaxed sub-problems to our initial problem.\\
\begin{theorem}
\begin{align}
\underset{\omega \in \Omega}{\max~}f_{\mathcal{P}}(\omega) = \underset{\mathcal{U} \subset \mathcal{P}}{\max~} \underset{\omega \in \Omega}{\max~} f_{\mathcal{U},r} (\omega)
\end{align}
and if we call this value $f^*$, $\forall \omega' \in \Omega$
\begin{align}
f_{\mathcal{P}}(\omega') = f^*
\Leftrightarrow
\exists \mathcal{U} \subset \mathcal{P},~ f_{\mathcal{U},r} (\omega') = f^*
\end{align}
\end{theorem}
\begin{proof}[Proof of the equality]
$\forall \mathcal{U} \subset \mathcal{P}$,
\begin{align*}
f_{\mathcal{P}} \geq f_{\mathcal{U}} \geq f_{\mathcal{U},r}
\end{align*}
which gives us the first inequality:
\begin{align*}
\underset{\omega \in \Omega}{\max~}f_{\mathcal{P}}(\omega) \geq \underset{\mathcal{U} \subset \mathcal{P}}{\max~} \underset{\omega \in \Omega}{\max~} f_{\mathcal{U},r} (\omega)
\end{align*}

Now, let us define $\forall \mathbf{\omega} \in \Omega$,~
$\mathcal{V}(\omega) = \{P \in \mathcal{P}, b_P(\omega) >0\}$
\\
We have $\forall \omega \in\Omega$,
\begin{align}
f_{\mathcal{P}}(\omega) = f_{\mathcal{V}(\omega)}(\omega) = f_{\mathcal{V}(\omega),r}(\omega)
\label{eq:vegal}
\end{align}
which, applied in $\omega_{\mathcal{P}}^*$, proves the equality
\end{proof}
\begin{proof}[Proof of the equivalence]
$\forall \omega \in \Omega$, $\forall \mathcal{U} \in \mathcal{P}$,
\begin{align*}
f_{\mathcal{U},r}(\omega) = f^* \Rightarrow
f_{\mathcal{V}(\omega),r}(\omega) = f^*
\end{align*}
because $f_{\mathcal{U},r}(\omega) \leq f_{\mathcal{V}(\omega),r}(\omega)$. And then 
\begin{align*}
f_{\mathcal{U},r}(\omega) = f^* \Rightarrow
f_{\mathcal{P}}(\omega) = f^*\\
\end{align*}
because of (\ref{eq:vegal}). The other implication is an immediate consequence of (\ref{eq:vegal}).
\end{proof}

Now, we cannot try to solve all the linear programs associated to the subset of $\mathcal{P}$ because their number is exponential($\Theta(2^{4n^{2}})$), but we can formulate a bilinear problem that will lead us to the solution.\\
Looking for
\begin{align} \max \limits_{\underset{\mathcal{U} \subset P}{\mathbf{\omega} \in \Omega}} f_{\mathcal{U},r} (\mathbf{\omega}) \end{align}
is the same as looking for
\begin{align} \max \limits_{\underset{\mathbf{\alpha} \in \{0,1\}^{|\mathcal{P}|}}{\mathbf{\omega} \in \Omega}} \sum_{P \in \mathcal{P}} \alpha_P w_P b_P(\omega) \end{align}
$\alpha_P$ representing whether $P$ belongs to $\mathcal{U}$.\\
This formulation is a Mixed Binary Bilinear Program.